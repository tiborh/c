\documentclass[12pt,a4paper,english,twoside]{article}
\usepackage{pslatex}
%\usepackage[T1]{fontenc}
%\usepackage[latin1]{inputenc}
%\usepackage{ucs}
%\usepackage[utf8x]{inputenc}
\usepackage[utf8]{inputenc}
\usepackage[T1]{fontenc}
\usepackage{a4wide}
\usepackage{babel}
\usepackage{hyperref}
%\usepackage{fancyhdr}
%\usepackage[dvips]{graphicx,color}
\usepackage{CJK}

\begin{document}
\begin{CJK}{UTF8}{min}
\begin{tabular}{p{1in}p{2in}p{3in}}
\bf 漢字&\bf 平仮名&\bf 英語\\\hline\hline\hline
関数&かんすう&function\\\hline
関&カン、せき、-ぜき、かか.わる、からくり、かんぬき&connection;  barrier;  gateway;  involve;  concerning\\\hline
数&スウ、ス、サク、ソク、シュ、かず、かぞ.える、しばしば、せ.める、わずらわ.しい&number;  strength;  fate;  law;  figures\\\hline\hline
内&うち&1. inside, within\\\hline\hline
警告&けいこく&waringin, advice\\\hline
警&ケイ、いまし.める&admonish;  commandment\\\hline
告&コク、つ.げる&revelation;  tell;  inform;  announce\\\hline\hline
書式&しょしき&bland form, format\\\hline
書&ショ、か.く、-が.き、-がき&write\\\hline
式&しき&style;  ceremony;  rite;  function;  method;  system;  form;  expression\\\hline\hline
対応&たいおう&interaction;  correspondence;  coping with;  dealing with;  support\\\hline
対&タイ、ツイ、あいて、こた.える、そろ.い、つれあ.い、なら.ぶ、むか.う&vis-a-vis;  opposite;  even;  equal;  versus;  anti-;  compare\\\hline
応&オウ、ヨウ、-ノウ、あた.る、まさに、こた.える&apply;  answer;  yes;  OK;  reply;  accept\\\hline\hline
引数が予期されます&ひきすうがよきされます&Arguments are expected\\\hline
引数&ひきすう&argument\\\hline
予期&よき&expectation;  assume will happen;  forecast\\\hline
される&される&する for passive\\\hline\hline
\end{tabular}

\end{CJK}
\end{document}
